%%%%%%%%%%%%%%%%%%%%%%%%%%
%%%%% MASTER PREAMBLE %%%% 
%%%%%%%%%%%%%%%%%%%%%%%%%%

% Basics 
\usepackage[utf8]{inputenc}
\usepackage[T1]{fontenc}
\usepackage{textcomp}
\usepackage[left=1in, right=1in, top=1in, bottom=1in]{geometry}

\usepackage{amsmath,amsfonts,amsthm,amssymb,mathtools}
\usepackage[varbb]{newpxmath}
\usepackage{xfrac}
\usepackage[usenames,dvipsnames]{xcolor} % usenames, dvipsnames adds more colours
\usepackage{hhline}
\usepackage{comment}
\usepackage{tasks}
\usepackage{enumerate} 
\usepackage{enumitem} 
\usepackage{titlesec}
\usepackage[most]{tcolorbox}
\usepackage{lipsum}
\usepackage{tabularx}
\usepackage{caption}
\usepackage{subfig}
\usepackage{pgfplots}
\usepackage{cancel} 
\usepackage{physics} 
\usepackage[bookmarks]{hyperref}
\usepackage{array}
\usepackage{float}

% Tables 
\numberwithin{table}{section}

% Inkscape Figures
\usepackage{import}
\usepackage{xifthen}
\usepackage{pdfpages}
\usepackage{transparent}
\newcommand{\incfig}[2][1]{
    \def\svgwidth{#1\columnwidth}
    \import{./figures/}{#2.pdf_tex}
}

\pdfsuppresswarningpagegroup=1

% Chemistry
\usepackage{lewis} 
\usepackage{bohr}
\usepackage[version=4]{mhchem}

% Page setup
\hypersetup{hidelinks}
\pagenumbering{arabic}
\pagestyle{plain}
\setlength{\parindent}{0pt}

% Show subsubsections
\setcounter{tocdepth}{3}
\setcounter{secnumdepth}{3}

% Required for the Grid
\usetikzlibrary{calc}

% Section Font-Size
\titleformat{\subsubsection}
  {\normalfont\fontsize{12}{12}\bfseries}{\thesubsubsection}{1em}{}

\titleformat{\subsection}
  {\normalfont\fontsize{14}{14}\bfseries}{\thesubsection}{1em}{}

\titleformat{\section}
  {\normalfont\fontsize{16}{16}\bfseries}{\thesection}{1em}{}

% Section Spacing
\titlespacing{\section}{0em}{2.5em}{1em}
\titlespacing{\subsection}{0em}{2.5em}{1em}
\titlespacing{\subsubsection}{0em}{2.5em}{1em}

% TABLE COLUMN SEPARATION (USES ARRAY PACKAGE)
% \renewcommand{\arraystretch}{1.8} % changes vertical space for each cell 
% \setlength{\tabcolsep}{18pt} % changes horizontal space for each cell
% \setlength{\arrayrulewidth}{0.25mm}

% TCOLORBOX 
% \newtcolorbox[auto counter, number within=section]{definition}{colback=white,title=Example~\thetcbcounter,breakable,colframe=white,boxrule=0pt, enhanced, title style={left color=gray!60,right color=white,middle color=white},arc=0mm, titlerule=0pt, fonttitle=\bfseries\sffamily}

% Theorems 
\usepackage{thmtools}
\usepackage[framemethod=TikZ]{mdframed}

\declaretheoremstyle[
    headfont=\bfseries\sffamily\color{ProcessBlue!70!black}, bodyfont=\normalfont,
    headpunct= :,
    mdframed={
        linewidth=2pt,
        rightline=false, topline=false, bottomline=false,
        linecolor=ProcessBlue, backgroundcolor=ProcessBlue!5,
        innerbottommargin=10pt
    } ]{note}

\declaretheoremstyle[
    headfont=\bfseries\sffamily\color{NavyBlue!70!black}, 
    bodyfont=\normalfont,
    headpunct=,
    mdframed={
        linewidth=2pt,
        rightline=false, topline=false, bottomline=false, linecolor=NavyBlue, innerbottommargin=10pt
    }
]{solution}

\declaretheoremstyle[
    headfont=\bfseries\sffamily\color{Gray!70!black}, bodyfont=\normalfont,
    headpunct= ,
    postheadspace=\newline,
    mdframed={
        linewidth=2pt,
        rightline=false, topline=false, bottomline=false,
        linecolor=Gray, backgroundcolor=Gray!5,
        innerbottommargin=10pt
    } ]{remark}

\declaretheoremstyle[
    headfont=\bfseries\sffamily\color{Fuchsia!70!black}, bodyfont=\normalfont,
    headpunct= ,
    mdframed={
        linewidth=2pt,
        rightline=false, topline=false, bottomline=false,
        linecolor=Fuchsia, backgroundcolor=Fuchsia!5,
        innerbottommargin=10pt
    }
]{example}

\declaretheoremstyle[
    headfont=\bfseries\sffamily\color{Fuchsia!70!black}, 
    bodyfont=\normalfont,
    headpunct= ,
    mdframed={
        linewidth=2pt,
        rightline=false, topline=false, bottomline=false,
        linecolor=Fuchsia,
    }
]{examplesolution}

\declaretheoremstyle[
    headfont=\bfseries\sffamily\color{black!70!black}, 
    bodyfont=\normalfont,
    headpunct=,
    postheadspace=\newline,
    mdframed={
        linewidth=2pt,
        rightline=false, topline=false, bottomline=false,
        linecolor=black,
    }
]{definition}

\declaretheorem[style=solution, name=Solution, numbered=no]{solution}

\declaretheorem[style=definition, name=Definition, numberwithin=section]{definitionswap}
\newenvironment{definition}[1]{\begin{definitionswap}[#1]}{\end{definitionswap}\vspace{0.5em}}

\declaretheorem[style=note, name=Note, numbered=no]{noteswap}
\newenvironment{note}[1]{\vspace{0.5em}\begin{noteswap}[#1]}{\end{noteswap}\vspace{0.5em}}

\declaretheorem[style=remark, name=Remark, numbered=no]{remarkswap}
\newenvironment{remark}[1]{\vspace{0.5em}\begin{remarkswap}[#1]}{\end{remarkswap}\vspace{0.5em}}

\declaretheorem[style=example, name=Example, numbered=no]{exampleswap}
% \newenvironment{example}{\vspace{0.5em}\begin{exampleswap}}{\end{exampleswap}}

\declaretheorem[style=examplesolution, name=Solution, numbered=no]{examplesolutionswap}
\newenvironment{examplesolution}{\vspace{-2em}\begin{examplesolutionswap}}{\end{examplesolutionswap}}

% Enumerate environments 
\newenvironment{2qu}
{
\begin{enumerate}[label=(\alph*)]
}
{\end{enumerate}}

\newenvironment{3qu}
{
\begin{enumerate}[label=(\roman*)]
}
{\end{enumerate}}

% Normal Environments 
\newenvironment{list0.5}
{
\begin{enumerate}
\setlength\itemsep{0.5em}
}
{\end{enumerate}}

% thmtools Environments
\newenvironment{problems}
{
    \subsection{Problems}
    \begin{enumerate}
    \setlength\itemsep{1em}
}
{\end{enumerate}}

\newenvironment{+problems}
{
    \subsubsection{Problems}
    \begin{enumerate}
    \setlength\itemsep{1em}
}
{\end{enumerate}}

\newenvironment{example}[1]
{
    \begin{exampleswap}
        #1
    \end{exampleswap}
    \begin{examplesolution}
}
{
\end{examplesolution}}

\newenvironment{2example}[1]
{
    \begin{exampleswap}[#1]
}
{
\end{exampleswap}}
